\thispagestyle{empty}
\gr{}
\begingroup
 
\begin{mdseries}
\hfill\break 
\textbf{\Large Ευχαριστίες}
\hfill\break

Η παρούσα διπλωματική εργασία εκπονήθηκε στο Εργαστήριο Υπολογιστικών Συστημάτων της Σχολής Ηλεκτρολόγων Μηχανικών και Μηχανικών Υπολογιστών του Εθνικού Μετσόβιου Πολυτεχνείου, υπό την επίβλεψη του καθηγητή Νεκτάριου Κοζύρη.

Θα ήθελα να ευχαριστήσω τον επιβλέποντα καθηγητή Νεκτάριο Κοζύρη για την ευκαιρία που μου έδωσε να ασχοληθώ με ένα σύγχρονο και ενδιαφέρον θέμα, καθώς και για τις γνώσεις που μου μετέδωσε μέσα από τα μαθήματα του. Επίσης, θέλω να ευχαριστήσω τη μεταδιδακτορικό ερευνήτρια Κατερίνα Δόκα για τη συνεχή καθοδήγηση και βοήθεια κατά τη διάρκεια της διπλωματικής εργασίας αλλά και το διδακτορικό φοιτητή Γιάννη Γιαννακόπουλο για την αμέριστη βοήθεια σε πρακτικά ζητήματα σε όλη την πορεία εκπόνησης διπλωματικής εργασίας.

Θα ήθελα ακόμη να ευχαριστήσω κάποιους ανθρώπους που άφησαν το στίγμα τους στα φοιτητικά μου χρόνια, ο καθένας με τον δικό του ξεχωριστό τρόπο. Θα αναφέρω τα ονόματα ορισμένων ανθρώπων τα οποία αργά ή γρήγορα θα κατέληγαν σε κάποιο σύγγραμα μου, αλλά και κάποια άλλα ονόματα των οποίων η σοφία θα γεμίζει πολλά.

Αρχικά, οφείλω να ευχαριστήσω τους καθηγητές της σχολής, Νικόλαο Παπασπύρου, Δημήτρη Φωτάκη, Ευστάθιο Ζάχο και Κώστα Κοντογιάννη, οι οποίοι όχι μόνο μου έδωσαν πολύτιμες γνώσεις αλλά μου μετέδωσαν το πάθος τους για την επιστήμη των υπολογιστών και υπήρξαν πρότυπα για μένα, τόσο ως μηχανικοί όσο και ως άνθρωποι.

Στη συνέχεια, θέλω να ευχαριστήσω το Μαρίνο Πατίρη που με στήριξε σε όλη την διάρκεια της διπλωματικής, που με βοηθάει σε ό,τι χρειαστώ και είναι πάντοτε δίπλα μου.

Ένα ειδικό ευχαριστώ στο φίλο μου Άλεξ. Π. Νάτσιο, που με μύησε στον κόσμο του \en FOSS\gr{} αλλά και στη φίλη μου Κατερίνα Μαυρομουστακάκη, που με το ταλέντο της δημιούργησε το \en trademark\gr{} του \en SecureWilly\gr{} και το εξώφυλλο της διπλωματικής.

Επιπλέον, θέλω να ευχαριστήσω τους φίλους και συμφοιτητές μου για τις όμορφες στιγμές που περάσαμε μαζί όλα αυτά τα χρόνια. Την Αντριάνα Δημητρίου, που ξενυχτούσαμε μαζί σε κάθε εργασία και κάθε εξεταστική, τον Αλέξανδρο Γκαβά, τον Χάρη Καλαντζή και τον Παύλο Καπούτση, που είμαστε μαζί από την πρώτη μέρα στη σχολή μέχρι και την τελευταία και ιδιαίτερα την φίλη μου Ρούλα Κυριακοπούλου, που βγάλαμε μαζί τα περισσότερα εργαστήρια της σχολής και με συμπλήρωνε πάντοτε άψογα. Θέλω επίσης να ευχαριστήσω τη Μαρία Λεονταρίδου, μια πολύτιμη φίλη που απέκτησα χάρη στη σχολή και έκανε κάθε πρωινό πιο ευχάριστο, αλλά και την κυρία Μαργαρίτα που πάντοτε πίστευε στις δυνατότητες μου. Ένα μεγάλο ευχαριστώ και στη Μαριλένα Μάρτο, που μαζί εκπληρώνουμε σιγά σιγά τα παιδικά μας όνειρα. Τέλος, ευχαριστώ τον φίλο και μέντορα Παντελή Σαράντο, που εκτίμησε τη δουλειά μου, πίστεψε σε μένα και μου έδωσε την ευκαιρία να ξεκινήσω ένα νέο ταξίδι.

Τίποτα από αυτά όμως δεν θα γινόταν πραγματικότητα χωρίς την στήριξη της οικογένειας μου. Ευχαριστώ λοιπόν, τον πατέρα μου που θα έπρεπε να είναι μηχανικός υπολογιστών αφού τις αυτοδίδακτες γνώσεις του θα ζήλευαν πολλοί κάτοχοι αυτού του τίτλου, τη μητέρα μου που όλη μου τη ζωή μου παρείχε εφόδια και αξίες, τον αδερφό μου Φώτη που βρίσκει τη λύση σε κάθε μου αδιέξοδο, τη θεία μου Ελένη που με βοηθάει πάντοτε σε ό,τι χρειαστώ και φυσικά τον \en Gizmo\gr{}, που κάνει την καθημερινότητα μου πιο όμορφη.
\end{mdseries}
\endgroup
\en
