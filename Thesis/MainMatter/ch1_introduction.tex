\chapter{Introduction}

\section{Motivation}

Today, we are encountering virtualization in most of our computing environments. This derives from the fact that one can isolate completely the runtime environment, thus keeping the host machine intact, which is highly beneficial for software development. Moreover, in the web development world, virtualization is a \say{must-have} which enables companies to optimize server operation costs. \cite{virtualdocker}

Docker has revolutionized virtualization, as it made it possible to package an application with all of its dependencies into a lightweight container. The virtualization that Docker performs is called operating-system-level virtualization or container-based virtualization, since the guests implemented are also named containers. Regardless of the recent overnight success and the explosive growth of Docker, containers is a preexisting feature, but their use for easily deploying applications was a new aspect imported by Docker. Nowadays, Docker is the most popular container standard, as it augments this type of container-based virtualization, introducing some useful novelty concepts, like descriptive configuration files and the capability to commit one's updates on a container. 

Docker became prominent, mainly due to its speed and portability. In contrast with full hardware virtualization (like VMware ESXi, or QEMU), operating-system-level virtualization comes with lighter overhead, compared with full hardware virtualization. Since the containers do not require an operating system boot, they start in less than a second and the performance is very near bare metal (direct / non-virtualized) performance. As for the portability, a container wraps up an application with everything it needs to run, like configuration files and dependencies. This enables an easy and reliable run of applications on different environments and no matter how complex the applications are, they can be containerized.
 
In the light of the above, it is evident that these advantages are the main reason why companies, with well known names included in them such as Paypal, Visa, Ebay, Netflix, Yelp, Spotify etc, are adopting Docker at a remarkable rate.

The other side of the coin, though, is that, if docker containers are not used wisely and secured, it is more easier for threats and exploits to make their appearance, than it is in VMs. It is safe to say that VM's are more secure, since containers make system calls directly to the Kernel. This leads to an extended set of vulnerabilities, especially in the matter of isolation.

Despite all the advantages of Docker, isolation is a compromise. While it is entirely possible to isolate Docker containers like VMs, most standard Docker containers, meaning those running on a basic community or commercial Docker Engine on Linux, are not isolated from each other like VMs.

In the current thesis, we address the concern that arises regarding docker's isolation by securing docker containers via Mandatory Access Control (MAC). The MAC system we focus on is AppArmor. AppArmor is a Linux security module (LSM), which means it is a kernel enhancement that protects an operating system and its applications from security threats, by confining programs to a limited set of resources with the usage of profiles. We developed a software that creates AppArmor profiles for docker services, which are adjusted to the task of each application, respecting the principal of least privilege, in order to preserve isolation, by restricting a container's allowed actions.

\section{Contribution}
The main contributions of this work are the following:
\begin{enumerate}
\item Design and implementation of an open source software, SecureWilly\footnote{Code available at \url{https://github.com/FaniD/SecureWilly}}, that creates profiles for any application, either it is single service or multi-service, in order to secure the containers and preserve the isolation.
\item While other programs that create AppArmor profiles exist, SecureWilly is the first program that handles multi-service projects and produces one profile for each service, considering the cooperation of the services.
\item Extensive research on the vulnerable features of docker that can lead to attacks and thorough analysis of each one of them.
\item Several examples of breakout container attacks are implemented, in the context of ethical hacking, in order to assist security.
\item Alerting user about the vulnerabilities detected in the docker project that could lead to an attack, such as privileged mode or entering host's namespaces.
\item Creation of AppArmor profiles for an instance of Nextcloud platform (two profiles were created, one for the application of Nextcloud and one for the database that it uses), as experimental evaluation of SecureWilly.
\end{enumerate}

\section{Chapter outline}
In the next section of \textbf{Chapter 1}, we describe briefly the main characteristics of SecureWilly, the software that we created, and the phases of its development.

\textbf{Chapter 2}  focuses on Containerization, Docker and the security tools that exist in order to protect it.

\textbf{Chapter 3} describes the development of SecureWilly, in order to automatically produce secure and efficient AppArmor profiles for every service of a docker project. 

\textbf{Chapter 4} studies the attacks that can be committed to containers, especially when they are relevant to the violation of the isolation between host and containers. Several techniques that can be used to commit such attacks are described and is explained how SecureWilly can be used in order to prevent them, either by adding some rules in the AppArmor profile or by providing alerts to the user.

\textbf{Chapter 5} shows the results of SecureWilly's usage on CloudSuite's benchmarks and Nextcloud and the evaluation of SecureWilly's functionality, performance and scalability is investigated. SecureWilly's profiles are compared to the respective genprof profile. AppArmor overhead is calculated by counting time on one of the implemented examples.

\textbf{Chapter 6} summarizes the main conclusions of the current thesis, shows related existing software and gives recommendations for future work. 

\section{Brief description of SecureWilly}
SecureWilly is the open source software we created in order to automatically produce AppArmor profiles for docker projects.

It handles both single service and multi-service projects and it respects the cooperation of services which is reflected on the rules of the AppArmor profiles.

Profiles are created per container and they follow the \say{Principle of Least Privilege}. This principle requires that in a particular abstraction layer of a computing environment, every module (such as a process, a user, or a program, depending on the subject) must be able to access only the information and resources that are necessary for its legitimate purpose. \cite{polp} This assures that each profile will restrict in the greater extent possible the corresponding service and it will allow exclusively the necessary operations of its task, while it will forbid any redundant action. Therefore, the profiles produced are secure and will defend isolation between host and containers.

Except for SecureWilly's main goal of producing AppArmor profiles, several other useful assets are also produced about the given docker project, such as alerts about the vulnerabilities detected, yml files for each service in case a docker-compose file does not exist and graphs illustrating the behaviour of each service through the rules of the profile produced.

The development of SecureWilly is divided in two phases:
\begin{itemize}
\item In the first phase, two parsers were used in two different types of analysis, in order to extract rules for the profile. Static analysis and its parser handles any initiative code of the docker project, given by the user and produces a preliminary profile, containing a minimum set of extracted rules from the code. Then, dynamic analysis takes place and its parser receives the preliminary profile of static analysis and uses it to exercise the docker project and extracts new rules by monitoring system logs.
\item In the second phase, we use reverse engineering by commiting container breakout attacks, in the context of ethical hacking, in order to create rules that would prevent these attacks and secure docker containers. This resulted in adding some fixed rules in the preliminary profile as well as producing some alerts about the vulnerabilities detected.
\end{itemize}


